% {{{ Praeambel
\documentclass[a5paper,12pt]{scrartcl}
% {{{ Sprache und Kodierung
\usepackage[ngerman]{babel}
\usepackage[utf8]{inputenc}
\usepackage[T1]{fontenc}
\usepackage{lmodern}
\renewcommand{\familydefault}{\sfdefault}
\usepackage{microtype}
\usepackage{textcomp}
% }}}
% {{{ Seitenlayout
\usepackage{geometry}
\geometry{left=1.5cm, right=1.5cm, top=2cm, bottom=1.5cm}
\pagestyle{empty}
\usepackage{enumitem}
% }}}
% {{{ Mathepakete
\usepackage{mathtools}
\usepackage{siunitx}
\usepackage{ziffer}
% }}}
% {{{ Aufgabenpakete
\usepackage{tasks}
% }}}
% {{{ Grafikpakete
\usepackage{graphicx}
\usepackage{tikz}
\usepackage{pgf}
% }}}
% {{{ Kopf und Fußzeile
\usepackage[headsepline]{scrpage2}
\renewcommand*{\headfont}{\normalfont}
\pagestyle{scrheadings}
\clearscrheadfoot{}
\ihead{Klasse 5b / CLA}
\chead{\textbf{Übersicht Mathematik}}
\ohead{SJ 19/20}
% }}}
% }}}

\begin{document}
\subsection*{Noten}
\paragraph{schriftliche Note:}
\begin{itemize}
	\item 4 Klassenarbeiten
	\item Mehrere Tests. Die Tests zählen zusammen so viel wie eine Klassenarbeit.
\end{itemize}
\paragraph{mündliche Note:}
\begin{itemize}
	\item die Qualität von Beiträgen (nicht bei der Einführung von Inhalten)
	\item Vorstellen von Ergebnissen
	\item Bereitschaft zum mathematischen Arbeiten
	\item Fehler sind sehr wertvoll!
\end{itemize}
	Die schriftliche Note zählt doppelt so viel wie die mündliche Note.
\subsection*{Benötigte Materialien}
\begin{itemize}
	\item 2 karierte DinA4-Hefte mit beidseitig durchgezogenem Rand (Regelheft und Übungsheft)
	\item 1 kariertes DinA4-Heft mit breitem durchgezogenen Rand (Klassenarbeitsheft)
	\item Füller, Geodreieck, gespitzter Bleistift und Taschenrechner
	\item Buntstifte: rot, grün, blau und gelb
	\item Zirkel nur, wenn es vorher angekündigt wurde
\end{itemize}
\paragraph{E-Mail} a.claessens@rgg-hausach.de
	\flushright\includegraphics[width=2cm]{/home/ac/Bilder/Schildkroete.png}
\end{document}
