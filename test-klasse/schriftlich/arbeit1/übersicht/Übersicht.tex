% chktex-file 8
% {{{ Praeambel
\documentclass[a5paper,landscape]{scrartcl}
% {{{ Sprache und Kodierung
\usepackage[ngerman]{babel}
\usepackage[utf8]{inputenc}
\usepackage[T1]{fontenc}
\usepackage{textcomp}
\usepackage{lmodern}
\usepackage{microtype}
\renewcommand{\familydefault}{\sfdefault}
% }}}
% {{{ Seitenlayout
\usepackage{geometry}
\geometry{left=1.5cm, right=1.5cm, top=2cm, bottom=1.5cm}
\pagestyle{empty}
\usepackage{enumitem}
% }}}
% {{{ Mathepakete
\usepackage{mathtools}
\usepackage{amssymb}
\usepackage{siunitx}
\usepackage{ziffer}
% }}}
% {{{ Aufgabenpakete
\usepackage{tasks}
% }}}
% {{{ Grafikpakete
\usepackage{tabularx}
\usepackage{booktabs}
\usepackage{booktabs}
\usepackage{graphicx}
\usepackage{tikz}
\usepackage{pgf}
% }}}
% {{{ Kopf und Fußzeile
\usepackage[headsepline]{scrlayer-scrpage}
\renewcommand*{\headfont}{\normalfont}
\pagestyle{scrheadings}
\clearscrheadfoot{}
\ihead{Klasse 5}
\chead{\textbf{Übersicht K1}}
\ohead{}
% }}}
% }}}

\begin{document}
\noindent
\begin{center}
\begin{tabular}{llc}
	\textbf{Thema} &  \textbf{Aufgaben} & \checkmark / noch zu tun\\
	\toprule
	\textbf{Römische Zahlen}  & \\
	- in beide Richtungen umrechnen & S. 31/1 \& 2\\
	- Streichholzrätsel & S. 32/11\\
	\midrule
	\textbf{Zahlenstrahl - größer und kleiner}  & \\
	- Zahlenstrahl lesen und zeichnen & S. 7/3 bis 6\\
	- Größer- und Kleinerzeichen verwenden & S. 7/7 \& 8\\
	- Bereiche markieren & S. 8/13\\
	\midrule
	\textbf{Das Zehnersystem - Runden von Zahlen} & \\
	- Zahlenwörter bis Billiarde als Wort und in Ziffern & S. 10/3 bis 5\\
	- Auf- und Abrunden & S. 11/8 bis 11\\
	\midrule
	\textbf{Addieren - Subtrahieren - Multiplizieren} & \\
	- schriftliches Rechnen & S. 14/6; AH 6/2; S. 23/8\&9\\
	- Überschlagsrechnungen &  S. 15/13; S. 23/10\\
	- Text in Rechnung übersetzen & S. 14/4; AH 6/5; S. 22/5\\
\end{tabular}
\end{center}
\end{document}
