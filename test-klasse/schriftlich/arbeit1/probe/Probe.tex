% chktex-file 1
% chktex-file 36
% {{{ Praeambel
\documentclass[a5paper]{scrartcl}
% {{{ Sprache und Kodierung
\usepackage[ngerman]{babel}
\usepackage[utf8]{inputenc}
\usepackage[T1]{fontenc}
\usepackage{textcomp}
\usepackage{lmodern}
\usepackage{microtype}
\renewcommand{\familydefault}{\sfdefault}
% }}}
% {{{ Seitenlayout
\usepackage{geometry}
\geometry{left=1.5cm, right=1.5cm, top=2cm, bottom=1.5cm}
\pagestyle{empty}
\usepackage{enumitem}
% }}}
% {{{ Mathepakete
\usepackage{mathtools}
\usepackage{siunitx}
\usepackage{ziffer}
% }}}
% {{{ Aufgabenpakete
\usepackage{exsheets}
\usepackage{tasks}
% }}}
% {{{ Grafikpakete
\usepackage{graphicx}
\usepackage{tikz}
\usepackage{pgf}
% }}}
% {{{ Kopf und Fußzeile
\usepackage[headsepline]{scrlayer-scrpage}
\renewcommand*{\headfont}{\normalfont}
\pagestyle{scrheadings}
\clearscrheadfoot{}
\ihead{Klasse 5}
\chead{\textbf{1. KA --- Probeaufgaben}}
\ohead{\today}
\newcommand{\Rom}[1]{\uppercase{\expandafter\romannumeral#1}}
% }}}
% }}}

\begin{document}
\begin{enumerate}
	\item Schreibe als Dezimalzahl: a
		\begin{tasks}(4)
			\task \Rom{6}
			\task \Rom{9}
			\task \Rom{1900}
			\task \Rom{48}
		\end{tasks}
	\item Schreibe als römische Zahl:
		\begin{tasks}(4)
			\task 12
			\task 53
			\task 900
			\task 2019
		\end{tasks}
	\item 
		\begin{tasks}
			\task Trage die Zahlen 0, 5, 14, 8, 2, 18 auf einen Zahlenstrahl ein.
			\task Trage die Zahlen 35, 75, 20, 5, 25, 50, 110, 90, 85 auf einen Zahlenstrahl ein.
		\end{tasks}
	\item 
		Markiere auf dem Zahlenstrahl
		\begin{tasks}
			\task alle Zahlen, die kleiner als 3 sind (grün).
			\task alle Zahlen, die größer als 3 und kleiner als 7 sind (blau).
			\task alle Zahlen, die größer als 7 sind (orange).
		\end{tasks}
		\vspace{21pt}
\begin{tikzpicture}
  \draw[->] (0,0) -- (10.5,0);
  \foreach \x in {0,...,10}
    \draw (\x,0.1) -- (\x,-0.1) node [below] {\x};
\end{tikzpicture}
\item Runde:
	\begin{tasks}(2)
		\task 354 auf Zehner
		\task \SI{2345}{} auf Hunderter
		\task \SI{123999}{} auf Zehner
		\task \SI{123456}{} auf Zehntausender
	\end{tasks}
\item
	\begin{tasks}
		\task Gebe an, wie viele Nullen eine Billion hat.
		\task Schreibe einundzwanzigmillionendreihunderttausendvierundsechzig als Dezimalzahl.
	\end{tasks}
\item Berechne schriftlich. Führe zunächst eine Überschlagsrechnung durch.
	\begin{tasks}(3)
		\task $\SI{345}{}+\SI{2386}{}$
		\task $\SI{345673}{}-\SI{211128}{}$
		\task $\SI{14345}{}\cdot \SI{28}{}$
	\end{tasks}
\item Berechne:
	\begin{tasks}
		\task Multipliziere 12 und 18.
		\task Addiere das Produkt von 23 und 45 mit der Differenz von 100 und 3.
	\end{tasks}
\end{enumerate}
\end{document}
