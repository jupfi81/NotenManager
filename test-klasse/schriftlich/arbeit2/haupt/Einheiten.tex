% chktex-file 36
%! TEX root = ./main.tex

\begin{Aufgabe}{4}
	\begin{tasks}
		\task Auf seinen Streifzügen durch die Nachbarschaft legt der Kater Luci \SI{3}{\kilo\meter} zurück. Zuhause läuft
		er nur die \SI{7}{\meter} zum Sofa. Berechne in Metern, wie weit er dann gelaufen ist.
		\task Der \SI{8}{\centi\meter} große Frosch Jumpi sagt stolz: "`Ich kann das 13-fache meiner Körpergröße hoch
		springen."' Berechne, wie hoch er springen kann.
		\task Lisa, Tanja, Tobi und Laura müssen Zeitungen austeilen. Hierzu muss eine \SI{2}{\kilo\meter} lange Strecke
		zurückgelegt werden, die sie gerecht unter sich aufteilen. Berechne, wie weit dann jeder gehen muss.
	\end{tasks}
\end{Aufgabe}
\begin{Loesung}
	\begin{tasks}
		\task Ansatz $\SI{3}{\kilo\meter}+\SI{7}{\meter}$ \HalberPunkt
					Rechnung $=\SI{3007}{\meter}$ \HalberPunkt
		\task Ansatz $13\cdot\SI{8}{\centi\meter}$ \HalberPunkt
					Rechnung $=\SI{\inteval{13*8}}{\centi\meter}$ \HalberPunkt
		\task Ansatz $\SI{2}{\kilo\meter}$ \HalberPunkt
					Rechnung $=\SI{500}{\meter}$ \HalberPunkt
	\end{tasks}
	Antwortsätze: \Punkt
\end{Loesung}

\begin{Aufgabe}{2}
	Tina hat eine Umfrage in ihrer Klasse gemacht, was die Lieblingshobbys sind. Stelle das Ergebnis in einem
	Balkendiagramm dar.

	\begin{tabular}{ccccc}
		Sport & Lesen & Computerspiel & Musik & Hausaufgaben\\\midrule
		7 & 12 & 3 & 5 & 0
	\end{tabular}
\end{Aufgabe}
\begin{Loesung}
	Richtige Längen \HalberPunkt, Skalierung \HalberPunkt, Achsenbeschriftung \HalberPunkt, Diagrammtitel \HalberPunkt
\end{Loesung}
