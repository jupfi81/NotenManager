% chktex-file 44
% chktex-file 36
% {{{ Praeamble
\documentclass[
	full-,
	a5paper,
	Probe,
	%NurAntworten,
	]{KA} % {{{ documentclass
\usepackage[ngerman]{babel}
\usepackage[utf8]{inputenc}
\usepackage[T1]{fontenc}
\usepackage{textcomp}
\usepackage{lmodern}
\usepackage{microtype}
\geometry{left = 1.2cm, right=1.5cm, bottom=1.5cm}
\usepackage{sfmath}
\renewcommand{\familydefault}{\sfdefault}

\usepackage{tasks}
\settasks{after-item-skip=0pt}
\usepackage{tikz}
\usepackage{pgfplots}
\usepackage{pgfplotstable}
\pgfplotsset{compat=1.15}
\usepackage{mathtools}
\usepackage{siunitx}
\usepackage{tabularx}
\usepackage{booktabs}
\usepackage{xfp}
\usepackage[misc]{ifsym}
\usepackage{xparse}
\newcolumntype{C}{>{\centering\arraybackslash}X}
% }}}
\settitle{2. KA --- Probearbeit}
\setdate{13.01.2020}
\setklasse{5b}
% }}}
\begin{document}
\begin{Aufgabe*}{0} % schriftliches Rechnen
	Berechne schriftlich.
	\begin{tasks}(4)
		\task $145\cdot 3$
		\task $\inteval{324*7}:7$
		\task $473\cdot 377$
		\task $\inteval{155*13}:13$
	\end{tasks}
\end{Aufgabe*}
\begin{Loesung}
	\begin{tasks}(2)
		\task $145\cdot 3=\inteval{145*3}$
		\task $\inteval{324*7}:7=324$
		\task $473\cdot 377=\inteval{473*377}$
		\task $\inteval{155*13}:13=155$
	\end{tasks}
\end{Loesung}
\begin{Aufgabe*}{0} % 5er-System -> 10er-System
	Schreibe die Zahl im Dezimalsystem.
	\begin{tasks}(4)
		\task ${(13)}_5$
		\task ${(422)}_5$
		\task ${(340)}_5$
		\task ${(2043)}_5$
	\end{tasks}
\end{Aufgabe*}
\begin{Loesung} % 5er-System -> 10er-System
	\begin{tasks}(2)
		\task ${(13)}_5=\inteval{5+3}$
		\task ${(422)}_5=\inteval{4*25+2*5+2}$
		\task ${(340)}_5=\inteval{3*25+4*5}$
		\task ${(2043)}_5=\inteval{2*125+4*5+3}$
	\end{tasks}
\end{Loesung}
\begin{Aufgabe*}{0} % 10er-System -> 5er-System
	Schreibe die Zahl im 5er-System.
	\begin{tasks}(4)
		\task $9$
		\task $17$
		\task $35$
		\task $251$
	\end{tasks}
\end{Aufgabe*}
\begin{Loesung} % 10er-System -> 5er-System
	Schreibe die Zahl im 5er-System.
	\begin{tasks}(4)
		\task $9={(14)}_5$
		\task $17={(32)}_5$
		\task $35={(120)}_5$
		\task $251={(2001)}_5$
	\end{tasks}
\end{Loesung}
\begin{Aufgabe*}{0} % Umrechnen von Längeneinheiten
	Rechne in die angegebene Einheit um.
	\begin{tasks}(3)
		\task \SI{21}{\centi\meter} in \si{\milli\meter}
		\task \SI{21}{\deci\meter} in \si{\milli\meter}
		\task \SI{450000}{\milli\meter} in \si{\meter}
	\end{tasks}
\end{Aufgabe*}
\begin{Loesung} % Umrechnen von Längeneinheiten
	Rechne in die angegebene Einheit um
	\begin{tasks}(3)
		\task \SI{21}{\milli\meter} = \SI{210}{\milli\meter}
		\task \SI{21}{\deci\meter} = \SI{2100}{\milli\meter}
		\task \SI{450000}{\milli\meter} = \SI{450}{\meter}
	\end{tasks}
\end{Loesung}
\begin{Aufgabe*}{0} % Rechnen mit Einheiten
\begin{tasks}
	\task Eva sägt einen \SI{10}{\meter} langen Baum in \SI{5}{\centi\meter} dicke Scheiben. Berechne, wie viele
	Scheiben sie erhält.
	\task Eine 1-Cent-Münze ist \SI{1}{\milli\meter} dick.
	Berechne, wie groß ein Turm aus \num{5000000} Münzen ist. Gib
	das Ergebnis in einer passenden Einheit an.
	\task Stella will ein Bücherregal bauen. Dafür braucht sie 20 Bretter, die jeweils \SI{30}{\centi\meter} lang sind.
	Im Baumarkt gibt es nur \SI{2}{\meter} lange Bretter, die sie selbst zusägen muss. Berechne, wie viele Bretter Stella mindestens kaufen muss.
\end{tasks}
\end{Aufgabe*}
\begin{Aufgabe}{0} % Balkendiagramm

	\begin{minipage}{0.55\textwidth}
	Das Haus von Marcos Eltern ist gerade vollgestopft mit 50 Besuchern.
	Um sich für ein Abendessen zu entscheiden, soll Marco eine Umfrage machen, was es zu Essen geben soll. Stelle seine
	Strichliste (rechts) als Balkendiagramm dar.
	\end{minipage}
	\hfill
	\begin{minipage}{0.4\textwidth}
		\fbox{
	\begin{tabular}{ll}
		Pizza &\StrokeFive\StrokeFive\StrokeFive\StrokeFive\\ % 20
		Lasagne &\StrokeFive\StrokeThree\\ % 8
		Kartoffelsalat &\StrokeFive\\ % 5
		Rosenkohlauflauf & \\ % 0
		Curry mit Reis& \StrokeFive\StrokeFive\StrokeFive\StrokeTwo\\ % 17
	\end{tabular}
}
	\end{minipage}
\end{Aufgabe}
\begin{minipage}{\textwidth-4.1cm}
\begin{Loesung}
	\begin{tasks}
		\task
			$\SI{10}{\meter}:\SI{5}{\centi\meter}
				=\SI{1000}{\centi\meter}:\SI{5}{\centi\meter}
				=\num{\inteval{1000/5}}$\\
			Sie erhält \inteval{1000/5} Scheiben.
		\task 
			$\num{5000000}\cdot\SI{1}{\milli\meter}
				= \SI{5000000}{\milli\meter}
				= \SI{5}{\kilo\meter}$\\
			Der Turm ist \SI{500}{\meter} hoch.
			\task Aus einem langen Brett kann sie\\
			$\SI{2}{\meter}:\SI{30}{\centi\meter}=
			\SI{200}{\centi\meter}:\SI{30}{\centi\meter}= 6 \text{ R } \SI{20}{\centi\meter}$\\
			zurecht sägen.
			Sie braucht also mindestens 4 lange Bretter (3 wären zu wenig).
	\end{tasks}
\end{Loesung}
\end{minipage}
\begin{minipage}{5.2cm}
\begin{Loesung}
	\begin{tikzpicture}
		\node at (3.3, 0) {\small Essen};
		\node at (-0.5, 3.5) {\small Anzahl};
		\begin{axis}[
			title = Was essen wir zu Abend?,
			ybar,
			bar width=0.3cm,
			ymin = 0, ymax = 23,
			axis lines = left,
			width=4.5cm,
			height=5cm,
			enlarge x limits=0.2,
			xticklabel style = {rotate=90},
			symbolic x coords={
				Pizza,
				Lasagne,
				Kartoffelsalat,
				Rosenkohlauflauf,
				Curry mit Reis
			},
			xtick=data,
		minor tick num = 1, grid=both, thick,
		]
		\addplot coordinates {
			(Pizza, 20)
			(Lasagne, 8)
			(Kartoffelsalat, 5)
			(Rosenkohlauflauf, 0)
			(Curry mit Reis, 17)
		};
		\end{axis}
	\end{tikzpicture}
\end{Loesung}
\end{minipage}
\end{document}
