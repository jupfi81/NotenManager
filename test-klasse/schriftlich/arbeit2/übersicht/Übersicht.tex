% chktex-file 8
% {{{ Praeambel
\documentclass[a5paper,landscape]{scrartcl}
% {{{ Sprache und Kodierung
\usepackage[ngerman]{babel}
\usepackage[utf8]{inputenc}
\usepackage[T1]{fontenc}
\usepackage{textcomp}
\usepackage{lmodern}
\usepackage{microtype}
\renewcommand{\familydefault}{\sfdefault}
% }}}
% {{{ Seitenlayout
\usepackage{geometry}
\geometry{left=1.5cm, right=1.5cm, top=2cm, bottom=1.5cm}
\pagestyle{empty}
\usepackage{enumitem}
% }}}
% {{{ Mathepakete
\usepackage{mathtools}
\usepackage{amssymb}
\usepackage{siunitx}
\usepackage{ziffer}
% }}}
% {{{ Aufgabenpakete
\usepackage{tasks}
% }}}
% {{{ Grafikpakete
\usepackage{tabularx}
\usepackage{booktabs}
\usepackage{booktabs}
\usepackage{graphicx}
\usepackage{tikz}
\usepackage{pgf}
% }}}
% {{{ Kopf und Fußzeile
\usepackage[headsepline]{scrlayer-scrpage}
\renewcommand*{\headfont}{\normalfont}
\pagestyle{scrheadings}
\clearscrheadfoot{}
\ihead{Klasse 5}
\chead{\textbf{Übersicht K2}}
\ohead{}
% }}}
% }}}

\begin{document}
\noindent
\begin{center}
\begin{tabular}{llc}
	\textbf{Thema} &  \textbf{Aufgaben} & \checkmark{} / noch zu tun\\
	\toprule
	\textbf{schriftliches Rechnen}  & \\
	- Multiplizieren & S. 23\\
	- Dividieren & S. 29\\
	- Addieren und Multiplizieren &  \\ \\
	\midrule
	\textbf{Zahlensysteme}  & \\
	- Vom Dezimalsystem zum 5er-System& \\
	- Vom 5er-System zum Dezimalsystem & \\
	\midrule
	\textbf{Längeneinheiten} & \\
	- Umrechnen von Längeneinheiten & S. 10/3 bis 5\\
	- Rechnen mit Längeneinheiten & S. 11/8 bis 11\\
	\midrule
	\textbf{Diagramme darstellen} & \\
	- Balkendiagramm zeichnen & \\
	- Daten aus einem Balkendiagramm entnehmen & \\
\end{tabular}
\end{center}
So will ich mich auf die Klassenarbeit vorbereiten:
\end{document}
