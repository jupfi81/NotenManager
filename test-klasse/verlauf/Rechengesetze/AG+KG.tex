% {{{ Praeambel
\documentclass[parskip=half-,a5paper]{scrartcl}
% {{{ Sprache und Kodierung
\usepackage[ngerman]{babel}
\usepackage[utf8]{inputenc}
\usepackage[T1]{fontenc}
\usepackage{textcomp}
\usepackage{lmodern}
\usepackage{microtype}
\renewcommand{\familydefault}{\sfdefault}
% }}}
% {{{ Seitenlayout
\usepackage{geometry}
\geometry{left=1.5cm, right=1.5cm, top=2cm, bottom=1.5cm}
\pagestyle{empty}
\usepackage{enumitem}
% }}}
% {{{ Mathepakete
\usepackage{mathtools}
\usepackage{siunitx}
\usepackage{ziffer}
% }}}
% {{{ Aufgabenpakete
\usepackage{exsheets}
\usepackage{tasks}
% }}}
% {{{ Grafikpakete
\usepackage{graphicx}
\usepackage{tikz}
\usepackage{pgf}
% }}}
% {{{ Kopf und Fußzeile
\usepackage[headsepline]{scrlayer-scrpage}
\renewcommand*{\headfont}{\normalfont}
\pagestyle{scrheadings}
\clearscrheadfoot{}
\ihead{Klasse 5}
\chead{\textbf{2. Rechengesetze}}
\ohead{SJ 20/21}
% }}}
% }}}

\begin{document}
\begin{tasks}
	\task \textbf{Kommutativgesetz bei Summen (KG+)}

	Kommen in einem Rechenausdruck nur Pluszeichen vor, darf beliebig getauscht
	werden.

	Beispiel: $13+62+87 = 13+87+62 = 100 + 62 = 162$
	%
	\task \textbf{Assoziativgesetz bei Summen (AG+)}

	Kommen in einem Rechenausdruck nur Pluszeichen vor, dürfen Klammern beliebig
	gesetzt und weggelassen werden.

	Beispiel: $24+(74+71) = (24+76)+71 = 100 + 71 = 171$
	%
	\task \textbf{Kommutativgesetz bei Produkten (KG$\cdot$)}

	Kommen in einem Rechenausdruck nur Malzeichen vor, darf beliebig getauscht
	werden.

	Beispiel: $4\cdot 13 \cdot 5= 4 \cdot 5 \cdot 13 = 20 \cdot 13 = 260$
	%
	\task \textbf{Assoziativgesetz bei Produkten (AG$\cdot$)}

	Kommen in einem Rechenausdruck nur Malzeichen vor, dürfen Klammern beliebig
	gesetzt oder weggelassen werden.

	Beispiel: $2\cdot (5 \cdot 135)=(2\cdot 5)\cdot 135 = 10\cdot 135 =1350$
\end{tasks}

\textbf{\large Vorsicht!}

Die obigen Rechengesetze gelten nicht, wenn Minuszeichen oder Geteiltzeichen
vorkommen.
\begin{itemize}
	\item $23-7+3=19$ aber $23-3+7=27$
	\item $23-(3+7)=23-10=13$ aber $(23-3)+7=20+7=27$
	\item $12:4\cdot 3 = 3 \cdot 3 = 9$ aber $12 : 3 \cdot 4 = 4 \cdot 4 = 16$
	\item $12:(4 \cdot 3) = 12 : 12 = 1$ aber $(12:4)\cdot 3 = 3\cdot 3 = 9$
\end{itemize}

\flushright{\includegraphics[width=2.5cm]{/home/ac/Bilder/Schildkroete.png}}
\end{document}
