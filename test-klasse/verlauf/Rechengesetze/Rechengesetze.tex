% chktex-file 44
% chktex-file 36
% {{{ Praeamble
\documentclass[parskip=half-,a4paper]{scrartcl} % {{{ documentclass
\usepackage[ngerman]{babel}
\usepackage[utf8]{inputenc}
% }}}
% {{{ Seitenränder und Zeilenabstand
\usepackage{geometry}
\geometry{left=1.5cm, right=1.5cm, top=2cm, bottom=1.5cm}
%\usepackage[onehalfspacing]{setspace}
% }}}
% {{{ Mathepakete
\usepackage{mathtools}
\usepackage{eurosym}
\usepackage{siunitx}
\sisetup{locale=DE, per-mode=symbol}
\DeclareSIUnit{\sieuro}{\mbox{\euro}}
\usepackage{ziffer}
\usepackage{tasks}
% }}}
% {{{ Grafikpakete
\usepackage{graphicx}
\usepackage{tikz}
\usepackage{pgfplots}
\usepackage{xcolor}
\newcommand{\red}[1]{{\color{red}#1}}
\newcommand{\green}[1]{{\color{green}#1}}
\newcommand{\blue}[1]{{\color{blue}#1}}
\newcommand{\orange}[1]{{\color{orange}#1}}
% }}}
% {{{ Tabellenpakete
\usepackage{tabularx}
\usepackage{booktabs}
% }}}
% {{{ Kopf- und Fußzeile
\pagestyle{empty}
% }}}
% }}}
\begin{document}
\section{Kommutativ- und Assoziativgesetz}
\subsection{Regelheft}

\begin{tasks}
	\task \textbf{Kommutativgesetz bei Summen (KG+)}

	Kommen in einem Rechenausdruck nur Pluszeichen vor, darf beliebig getauscht
	werden.

	Beispiel: $13+62+87 = 13+87+62 = 100 + 62 = 162$
	%
	\task \textbf{Assoziativgesetz bei Summen (AG+)}

	Kommen in einem Rechenausdruck nur Pluszeichen vor, dürfen Klammern beliebig
	gesetzt und weggelassen werden.

	Beispiel: $24+(74+71) = (24+76)+71 = 100 + 71 = 171$
	%
	\task \textbf{Kommutativgesetz bei Produkten (KG$\cdot$)}

	Kommen in einem Rechenausdruck nur Malzeichen vor, darf beliebig getauscht
	werden.

	Beispiel: $4\cdot 13 \cdot 5= 4 \cdot 5 \cdot 13 = 20 \cdot 13 = 260$
	%
	\task \textbf{Assoziativgesetz bei Produkten (AG$\cdot$)}

	Kommen in einem Rechenausdruck nur Malzeichen vor, dürfen Klammern beliebig
	gesetzt oder weggelassen werden.

	Beispiel: $2\cdot (5 \cdot 135)=(2\cdot 5)\cdot 135 = 10\cdot 135 =1350$
\end{tasks}

\textbf{\large Vorsicht!}

Die obigen Rechengesetze gelten nicht, wenn Minuszeichen oder Geteiltzeichen
vorkommen.
\begin{itemize}
	\item $23-7+3=19$ aber $23-3+7=27$
	\item $23-(3+7)=23-10=13$ aber $(23-3)+7=20+7=27$
	\item $12:4\cdot 3 = 3 \cdot 3 = 9$ aber $12 : 3 \cdot 4 = 4 \cdot 4 = 16$
	\item $12:(4 \cdot 3) = 12 : 12 = 1$ aber $(12:4)\cdot 3 = 3\cdot 3 = 9$
\end{itemize}

\begin{tasks}
	\task \textbf{Plus- und Minusglieder zusammenfassen}

		Kommen Plus- und Minuszeichen vor, kann man die Plusglieder in einer Klammer
		und die Minusglieder in einer anderen Klammer sammeln.

		Beispiel: $\blue{30}-\red{19}-\red{3}+\blue{8}-\red{4} =
		(\blue{30}+\blue{8})-(\red{19}+\red{3}+\red{4})= \blue{38}-\red{26}=12$
\end{tasks}

\subsection{Ablauf}

\begin{itemize}
	\item Den Regelheftaufschrieb nachtragen.
	\item Übungsaufgaben rechnen: S. 107/6adg, 10abc, 12
\end{itemize}

\section{Distributivgesetz}
\subsection{Regelheft}
\subsubsection{Ausmultiplizieren}
$\red{3}\cdot (20+4) = \red{3}\cdot 20 + \red{3} \cdot 4$

$7\cdot (100-3) = 7\cdot 100 - 7 \cdot 3$

Anstatt die Klammer zuerst zu berechnen, kann man auch jeden Summanden mit dem
Faktor vor der Klammer multiplizieren und die Ergebnisse addieren.

\subsubsection{Ausklammern}
$\red{13}\cdot 7 + \red{13} \cdot 3$

Anstatt erst die Produkte zu berechnen, kann man zuerst den gemeinsamen Faktor
vor die Klammer schreiben. Dann wird die Klammer berechnet und dann
multipliziert.

\subsection{Verlauf}
\begin{itemize}
	\item Einstiegstext aus dem Buch lesen.
	\item Übungen S. 110/1abc, 4abc (6 Minuten)
	\item Übungens S. 111/11, 13, 15, 17, 21 (Selbstkontrolle)
\end{itemize}

\newpage
\section{Potenzen}
\subsection{Regelheft}

\begin{align*}
	\text{Potenz: } 5^3=3\cdot3\cdot3\cdot3\cdot3
\end{align*}
$5$ ist die Basis/Grundwert und $3$ ist die Hochzahl/Exponent
\begin{itemize}
	\item 10er Potenzen: Potenzen Mit Grundwert 10
		($10^2=100,~10^3=1000,~10^6=1000000$ also eine eins mit 6 Nullen)
	\item Quadratzahlen: Potenzen mit Exponent 2 ($2^2=4,~4^2=16,~18^2=324$)
	\item Kubikzahlen: Potenzen mit Exponent 3 ($2^3=8,~3^3=27,~5^3=125$)
	\item Rechenreihenfolge: Hoch vor Klammer vor Punkt vor Strich
		(\textbf{K}l\textbf{a}mmer vor \textbf{Ho}chzahl vor \textbf{P}unkt vor
		\textbf{S}trich $\to$ KaHops (Känguru hopst))
\end{itemize}
\subsection{Verlauf}
\begin{itemize}
	\item Einstieg:
	\begin{itemize}
		\item Das Eichhörnchen Peter sammelt jeden Tag 3 Nüsse mehr, als am Tag davor.

		\begin{tabular}{ccccc}
			1. Tag & 2. Tag & 3. Tag & 4. Tag & 5. Tag \\
			3 & 3+3 & 3+3+3 & 3+3+3+3 & $\underbrace{3+3+3+3+3}_{\text{5 3er werden
			addiert}} = 5 \cdot 3$
		\end{tabular}

		\item Das Eichhörnchen Clara sammelt jeden Tag dreimal soviele Nüsse, wie am Tag davor.

		\begin{tabular}{ccccc}
			1. Tag & 2. Tag & 3. Tag & 4. Tag & 5. Tag \\
			3 & $3\cdot3$ & $3\cdot3\cdot3$ & $3\cdot3\cdot3 \cdot3$ & $\underbrace{3\cdot3\cdot3\cdot3\cdot3}_{\text{5 3er werden
			multipliziert}} = 3^5$
		\end{tabular}
\end{itemize}
	\item Regelheftaufschrieb Potenzen
	\item einfache Übungsaufgaben (S. 114; 1,2)
	\item Begriffe einführen:
		\begin{itemize}
			\item Quadratzahlen wiederholen (kennt ihr ja schon)
			\item 10er-Potenzen (berechne $10^2$, $10^2$, $10^5$, $10^2$)
		\end{itemize}
	\item einfache Übungsaufgaben + Rechenreihenfolge besprechen:
		(\textbf{K}l\textbf{a}mmer vor \textbf{Ho}chzahl vor \textbf{P}unkt vor
		\textbf{S}trich $\to$ KaHops (Känguru hopst))
	\item längere Übungsaufgaben
\end{itemize}
\section{Teilbarkeitsregeln}
\subsection{Regelheft}
\subsubsection{Teilbarkeit}
6 ist ein Teiler von 4, da
\begin{itemize}
	\item 24 durch 6 ohne Rest teilen kann.
	\item 24 ein Vielfaches von 6 ist.
\end{itemize}
Die Teiler von 24: $24 = 1\cdot 24 = 2 \cdot 12 = 3 \cdot 8 = 4\cdot 6$ Die
Teiler sind
$1;~2;~3;~4;~6;~8;~12;~24$\\
Die Vielfachen von 26 kleiner als 200:
$1\cdot 26 = 26;~2\cdot 26 = 52;~ 3\cdot 26=78;~4\cdot 26=108$

\subsubsection{Endstellenregel}
\begin{enumerate}
	\item Eine Zahl ist durch 10 teilbar, wenn ihre letzte Ziffer eine 0 ist.
		\begin{align*}
			23650 = 2365\cdot 10
		\end{align*}
	\item Eine Zahl ist durch 10 teilbar, wenn ihre letzte Ziffer eine 0 oder 5
		ist,
		\begin{align*}
			23650=(2365\cdot 2) \cdot 5
			\qquad
			23455 = (2345\cdot 2 + 1)\cdot 5
		\end{align*}
	\item Eine Zahl ist durch 2 teilbar, wenn sie die Endziffer 0, 2, 4, 6 oder 8
		hat.
		\begin{align*}
			4572 = 457 \cdot 10 + 2 = 
		\end{align*}
	\item Eine Zahl ist durch 4 teilbar, wenn ihre letzten beiden Ziffern durch 4
		teilbar sind.
	\item Eine Zahl ist durch 8 teilbar, wenn ihre letzten drei Ziffern durch 8
		teilbar sind.
\end{enumerate}
\subsection{Verlauf}
\begin{itemize}
	\item Regelhefteintrag \emph{Teilbarkeit} erarbeiten
	\item S.117/2,3,4
	\item 10 1-Cent Münzen auf den Tisch legen --- wie kann man die auf 3 Leute
		verteilen?
	\item Wie viele bleiben bei 40, 80, 70, 72, 64, 45 übrig?
	\item Und jetzt mit 1457 (mit Geld hinlegen)
	\item Quersummenregel festhalten (Regelheft).
	\item Restliche Übungsaufgaben.
	\item Eventuell noch Primzahlen einführen (Sieb von keine Ahnung wem), dann
		die Primfaktorzerlegung.
\end{itemize}
\end{document}
