% chktex-file 44
% chktex-file 3
% chktex-file 36
% {{{ Praeambel
\documentclass[parskip=half-,a4paper]{scrartcl}
% {{{ Sprache und Kodierung
\usepackage[ngerman]{babel}
\usepackage[utf8]{inputenc}
\usepackage[T1]{fontenc}
\usepackage{textcomp}
\usepackage{lmodern}
\usepackage{microtype}
\renewcommand{\familydefault}{\sfdefault}
% }}}
% {{{ Seitenlayout
\usepackage{geometry}
\geometry{left=1.5cm, right=1.5cm, top=2cm, bottom=1.5cm}
\pagestyle{empty}
\usepackage{enumitem}
% }}}
% {{{ Mathepakete
\usepackage{mathtools}
\usepackage{siunitx}
\usepackage{ziffer}
% }}}
% {{{ Aufgabenpakete
\usepackage{tasks}
\usepackage{tabularx}
\newcolumntype{Y}{>{\centering\arraybackslash}X}
% }}}
% {{{ Grafikpakete
\usepackage{graphicx}
\usepackage{tikz}
\usepackage{pgf}
% }}}
% {{{ Kopf und Fußzeile
\usepackage[headsepline]{scrlayer-scrpage}
\renewcommand*{\headfont}{\normalfont}
\pagestyle{scrheadings}
\clearscrheadfoot{}
\ihead{Klasse 5}
\chead{\textbf{Rechnen im 5er-System}}
\ohead{\today}
% }}}
% }}}

\begin{document}

\renewcommand{\arraystretch}{1.5}
% {{{ Addition
\begin{minipage}[t]{11cm}
Das 1+1 im 10er-System.
\par\bigskip

\begin{tabularx}{11cm}{Y||Y|Y|Y|Y|Y|Y|Y|Y|Y|Y}
	+ & 0 & 1 & 2 & 3 & 4 & 5 & 6 & 7 & 8 & 9 \\\hline\hline
	0&&&&&&&&&\\\hline
	1&&&&&&&&&\\\hline
	2&&&&&&&&&\\\hline
	3&&&&&&&&&\\\hline
	4&&&&&&&&&\\\hline
	5&&&&&&&&&\\\hline
	6&&&&&&&&&\\\hline
	7&&&&&&&&&\\\hline
	8&&&&&&&&&\\\hline
	9&&&&&&&&&\\
\end{tabularx}
\end{minipage}
\hfill
\begin{minipage}[t]{6cm}
Das 1+1 im 5er-System.
\par\bigskip

\begin{tabularx}{6cm}{Y||Y|Y|Y|Y|Y}
	+ & $(0)_5$ & $(1)_5$& $(2)_5$& $(3)_5$& $(4)_5$ \\\hline\hline
	$(0)_5$&&&&\\\hline
	$(1)_5$&&&&\\\hline
	$(2)_5$&&&&\\\hline
	$(3)_5$&&&&\\\hline
	$(4)_5$&&&&
\end{tabularx}
\end{minipage}

\textbf{Aufgabe 1}\par
Fülle die beiden Tabellen aus. Achte bei der rechten Tabelle darauf, dass es nur die Ziffern 0, 1, 2, 3 und 4 gibt.

\textbf{Aufgabe 2}\par
Stelle dir vor, du sollst jemandem erklären, der nicht Kopfrechnen kann, wie man zusammen mit der linken Tabelle
schriftlich addiert. Schreibe einen kleinen Text. Nehme als Beispiel die Rechnung $253+86$.

\textbf{Aufgabe 3}\par
Rechne die folgenden Aufgaben direkt im 5er-System mit der rechten Tabelle. Überprüfe anschließend, ob deine Rechnung
stimmt, indem du alle Zahlen ins 10er-System übersetzt.

\begin{tasks}(4)
	\task $(24)_5+(12)_5$
	\task $(33)_5+(22)_5$
	\task $(1000)_5+(2000)_5$
	\task $(3421)_5+(2342)_5$
	\task $(30)_5+(20)_5$
	\task $(22)_5+(44)_5$
	\task $(300)_5+(300)_5$
	\task $(4413)_5+(3024)_5$
\end{tasks}

\textbf{Aufgabe 5}\par
Schreibe eine Erklärung wie in Aufgabe 2 für die Addition im 5er-System.

% }}}
\newpage
% {{{ Multiplikation
\begin{minipage}[t]{11cm}
Das 1$\cdot$1 im 10er-System.
\par\bigskip

\begin{tabularx}{11cm}{Y||Y|Y|Y|Y|Y|Y|Y|Y|Y|Y}
	$\cdot$ & 0 & 1 & 2 & 3 & 4 & 5 & 6 & 7 & 8 & 9 \\\hline\hline
	0&&&&&&&&&\\\hline
	1&&&&&&&&&\\\hline
	2&&&&&&&&&\\\hline
	3&&&&&&&&&\\\hline
	4&&&&&&&&&\\\hline
	5&&&&&&&&&\\\hline
	6&&&&&&&&&\\\hline
	7&&&&&&&&&\\\hline
	8&&&&&&&&&\\\hline
	9&&&&&&&&&\\
\end{tabularx}
\end{minipage}
\hfill
\begin{minipage}[t]{6cm}
Das 1$\cdot$1 im 5er-System.
\par\bigskip

\begin{tabularx}{6cm}{Y||Y|Y|Y|Y|Y}
	$\cdot$ & $(0)_5$ & $(1)_5$& $(2)_5$& $(3)_5$& $(4)_5$ \\\hline\hline
	$(0)_5$&&&&\\\hline
	$(1)_5$&&&&\\\hline
	$(2)_5$&&&&\\\hline
	$(3)_5$&&&&\\\hline
	$(4)_5$&&&&
\end{tabularx}
\end{minipage}

\textbf{Aufgabe 1}\par
Fülle die beiden Tabellen aus. Achte bei der rechten Tabelle darauf, dass es nur die Ziffern 0, 1, 2, 3 und 4 gibt.

\textbf{Aufgabe 2}\par
Stelle dir vor, du sollst jemandem erklären, der nicht Kopfrechnen kann, wie man zusammen mit der linken Tabelle
schriftlich multipliziert. Schreibe einen kleinen Text. Nehme als Beispiel die Rechnung $253\cdot 24$.

\textbf{Aufgabe 3}\par
Rechne die folgenden Aufgaben direkt im 5er-System mit der rechten Tabelle. Überprüfe anschließend, ob deine Rechnung
stimmt, indem du alle Zahlen ins 10er-System übersetzt.

\begin{tasks}(4)
	\task $(24)_5\cdot(12)_5$
	\task $(33)_5\cdot(22)_5$
	\task $(1000)_5\cdot(2000)_5$
	\task $(3421)_5\cdot(2342)_5$
	\task $(30)_5\cdot(20)_5$
	\task $(22)_5\cdot(44)_5$
	\task $(300)_5\cdot(300)_5$
	\task $(4413)_5\cdot(3024)_5$
\end{tasks}

\textbf{Aufgabe 5}\par
Schreibe eine Erklärung wie in Aufgabe 2 für die Multiplikation im 5er-System.

% }}}

\end{document}
